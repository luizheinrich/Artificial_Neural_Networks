%=================================================================================================
%=							       		    INTRODUÇÃO							     			 =
%=================================================================================================

\section{Introdução} \label{introducao}

As Redes Neurais Artificiais (RNA) são modelos matemáticos-computacionais que visam simular o comportamento e arquitetura de um neurônio biológico. Essa característica confere a uma RNA a capacidade de reconhecer e classificar padrões a partir de um modelo de aprendizagem baseado no aprendizado humano \cite{kovacs2002redes}.

Fazendo o uso de RNAs, o presente artigo trata do problema de classificação de Vinho. E este, é um conjunto de dados multivariado introduzido por \textit{M. Florina}, o qual consiste em 178 amostras de 13 quesitos do vinho avaliado, pertencentes às três classes de vinho.

Estes dados são os resultados de uma análise química de vinhos cultivados na mesma região na Itália, no entanto, derivados de três diferentes cultivares. A análise determinou as quantidades de 13 constituintes encontrados em cada um dos três tipos de vinhos \cite{fraley2007model}.

Desta forma, com base na combinação dessas 13 características, é possível classificar o tipo de vinho com base nas 178 amostras.

\subsection{Metodologia}

Para o fim proposto, foi usado o Software Matlab (R2018B), com o uso da toolbox de redes neurais.


\subsection{Objetivos}

Como objetivo geral, este trabalho implementa duas arquiteturas de Redes Neurais Artificiais para a classificação de três classes de vinho a partir dos 13 parâmetros de entrada. A primeira arquitetura é uma RNA de Múltiplas Camadas (MLP) e a segunda, a uma MLP com estrutura Competitiva Auto Associativa.

Enquanto que o objetivo específico é averiguar e comparar a melhor arquitetura de rede para o problema de classificação proposto.


\subsection{Organização do Trabalho}

Este artigo está organizado como se segue. Na Seção \ref{revisaobibliografica} é feita uma revisão bibliográfica sobre Redes Neurais. Logo após, na Seção \ref{baseDados}, é apresentada a base de dados estudada neste artigo. Na Seção \ref{desenvolvimentoMLP} e na Seção \ref{desenvolvimentoAuto}, são apresentadas as arquiteturas das redes propostas, bem como por quais metodologias este trabalho se orienta. Os resultados são apresentados no Seção \ref{resultados}. Na Seção \ref{consideracoesFinais}, as considerações finais desde trabalho.