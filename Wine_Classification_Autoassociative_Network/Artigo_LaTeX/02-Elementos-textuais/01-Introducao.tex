%=================================================================================================
%=							       		    INTRODUÇÃO							     			 =
%=================================================================================================

\section{Introdução}

As Redes Neurais Artificiais (RNA) são modelos matemáticos-computacionais que visam simular o comportamento e arquitetura de um neurônio biológico. Essa característica confere a uma RNA a capacidade de reconhecer e classificar padrões a partir de um modelo de aprendizagem baseado no aprendizado humano.

O presente artigo trata do problema de classificação de Vinho. E este, é um conjunto de dados multivariado introduzido pelo estatístico e biólogo britânico \textit{Ronald Fisher} em seu artigo de 1936.

O conjunto de dados consiste em 50 amostras de cada uma das três espécies de Iris, a saber: \textbf{Iris Setosa}, \textbf{Iris Virginica} e \textbf{Iris Versicolor}. Quatro características foram medidas a partir de cada amostra: o comprimento e a largura das sépalas e pétalas, em centímetros. Com base na combinação dessas quatro características, Fisher desenvolveu um modelo discriminante linear para distinguir as espécies umas das outras.

Estes dados são os resultados de uma análise química de vinhos cultivados na mesma região na Itália, mas derivados de três diferentes cultivares. A análise determinou as quantidades de 13 constituintes encontrados em cada um dos três tipos de vinhos.

\subsection{Metodologia}

Para o fim proposto, foi usado o Software Matlab (R2018B), com o uso da toolbox de redes neurais.


\subsection{Objetivos}

Como objetivo geral, este trabalho implementa duas arquiteturas de Redes Neurais Artificiais para a classificação de três classes de vinho a partir dos 13 parâmetros de entrada. A primeira arquitetura é uma RNA de Múltiplas Camadas (MPL) e a segunda, a uma MPL com estrutura Competitiva Autoassociativa.

Enquanto que o objetivo específico é averiguar e comparar a melhor arquitetura de rede para o problema de classificação proposto.


\subsection{Organização do Trabalho}

Este artigo está organizado como se segue. No ceção 2 é apresentado a arquitetura de rede e por quais metodologias este trabalho se orienta. Os resultados são apresentados no capítulo 3. No capítulo 4, as considerações finais desde trabalho.