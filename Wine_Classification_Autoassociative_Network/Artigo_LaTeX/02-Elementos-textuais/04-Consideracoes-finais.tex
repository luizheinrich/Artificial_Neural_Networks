\section{Considerações Finais} \label{consideracoesFinais}

\subsection{Discussão}

A proposta de uma Rede Neural especializada em cada classe confere um desempenho muito bom em problemas de classificação. Assim, em uma estrutura Competitiva Auto Associativa, o que se tem são três redes dedicadas a cada uma das classes. Diferentemente da Rede MLP, onde uma só rede tem o dever de classificar as três classes de vinho estudada.


\subsection{Conclusão}

As redes com estrutura Competitiva Auto Associativa demonstraram um desempenho significativamente maior frente à rede MPL. Portanto, a partir dos estudos das arquiteturas de redes neurais envolvidas neste trabalho, concluímos que as redes Auto Associativas têm uma grande confiabilidade para problemas de classificação. No entanto, deve-se ressaltar, que a base de dados estudada também apresenta uma certa facilidade para a classificação, já que a própria rede MLP errou uma única vez na topologia de neurônios. Certamente, essa facilidade da base de dados pode justificar a eficácia de 100\% de acerto da rede auto associativa. Portanto, para problemas de classificação mais complexos, não necessariamente a eficácia de 100\% de acertos se repetirá para uma estrutura competitiva auto associativa.
